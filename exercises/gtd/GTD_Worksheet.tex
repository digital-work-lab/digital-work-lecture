\documentclass[11pt]{scrartcl}
%==============================
%ZUSÄTZLICH EINGEBUNDENE PAKETE
%==============================
\usepackage[T1]{fontenc}
\usepackage[utf8]{inputenc}
\usepackage[ngerman]{babel}
\usepackage{amsmath,amsthm,amssymb,euscript}
\usepackage{lmodern}
\usepackage{enumerate}
\usepackage{graphicx}
\usepackage{tabulary}
\usepackage{txfonts}
\usepackage{eurosym}
\usepackage{listings, color}
\definecolor{darkblue}{rgb}{0,0,.6}
\definecolor{darkred}{rgb}{.6,0,0}
\definecolor{darkgreen}{rgb}{0,.6,0}
\definecolor{red}{rgb}{.98,0,0}
\lstloadlanguages{Java}
\lstset{
basicstyle=\footnotesize\ttfamily,
showspaces=false,
showtabs=false,
columns=fixed,
numbers=left,
frame=none,
numberstyle=\tiny,
breaklines=true,
showstringspaces=false,
xleftmargin=1cm,
tabsize=4}

%tables
\usepackage{tabularx}

%circles in math mode
\usepackage{tikz}
\newcommand*\circled[1]{\tikz[baseline=(char.base)]{
            \node[shape=circle,draw,inner sep=2pt] (char) {#1};}}

%plotting
\usepackage{pgfplots}

\usepackage[a4paper]{geometry}
\geometry{a4paper,tmargin=3.5cm, bmargin=3cm, lmargin=3cm, rmargin=2cm, headheight=13em, headsep=3em, footskip=1cm}

\usepackage{textcomp}

\usepackage{fancyhdr}
\setlength{\parskip}{1em}
\setlength{\parindent}{0pt}

\newcommand{\blatt}[1]{\begin{center}{\bf\Large  #1 \vspace*{0.3cm}}\end{center}}

\newtheoremstyle{aufgabenstyle}% name of the style to be used
  {24pt}% measure of space to leave above the theorem. E.g.: 3pt
  {6pt}% measure of space to leave below the theorem. E.g.: 3pt
  {\normalfont}% name of font to use in the body of the theorem
  {}% measure of space to indent
  {\bfseries}% name of head font
  {}% punctuation between head and body
  {\newline}% space after theorem head; " " = normal interword space
  {\thmname{#1}\thmnumber{ #2}\thmnote{ (#3)}\newline}% Manually specify head

\theoremstyle{aufgabenstyle}
\newtheorem{aufgabe}{\large Aufgabe}

%==============================
%BEGINN DER KOPF- UND FUSSZEILE
%==============================
\pagestyle{fancy}
\fancyhf{}
\fancyhead[R]{Prof. Dr. Gerit Wagner}
\fancyhead[L]{Introduction to Digital Work}
\setlength{\parskip}{1em}
\setlength{\parindent}{0pt}
\fancyfoot[C]{\thepage}
%\fancyfoot[L]{\today}
%================================
%BEGINN DES EIGENTLICHEN INHALTES
%================================
\begin{document}
\begin {center}\textbf{{\Large Worksheet: Getting-Things-Done}}\\
\end{center}

\textbf{Goal}: Apply the \textbf{Getting Things Done (GTD)} method to help a hypothetical friend prepare for their first semester.


\textbf{\large{Task 1: Set up a System with the core GTD Components}}

\begin{itemize}
	\item Which lists should be created digitally or physically?
	\item What would be included on the list? Give examples.
\end{itemize}


\renewcommand{\arraystretch}{2.5} % Increase line height
\begin{table}[htbp]
	\centering
	\begin{tabular}{|p{3cm}|p{3cm}|p{3cm}|p{5cm}|}
		\hline
		\textbf{List} & \textbf{Digital} & \textbf{Physical} & \textbf{Content (example)} \\
		\hline
		 &  &  &  \\ \hline
		 &  &  &  \\ \hline
		 &  &  &  \\ \hline
		 &  &  &  \\ \hline
		 &  &  &  \\ \hline
		 &  &  &  \\ \hline
		 &  &  &  \\ \hline
		 &  &  &  \\ \hline


%		Inbox & Email inbox, Notes app & In-tray, notebook & Lecture email, to-do jotted on phone \\ \hline
%		Projects & Project board in Notion or Trello & Project section in planner & “Prepare for Data Science exam” \\
%		\hline
%		Action Lists & Todoist, Microsoft To Do & Sticky notes, checklist & “Read chapter 3,” “Email Prof. Smith” \\
%		\hline
%		Waiting-for & Email with self-CC or label & Sticky with names/tasks & “Waiting for team to share slides” \\
%		\hline
%		Calendar & Google Calendar, Outlook & Wall calendar, paper planner & Exam registration deadline \\
%		\hline
%		Reference & Cloud folders (Google Drive) & Binder or printed handouts & FlexNow User’s Guide, lecture slides \\
%		\hline
%		Checklists & Notion templates, Google Docs & Printed templates & “Checklist for exam day” \\
%		\hline
%		Tickler & Dated folders/reminders in app & 43 physical folders & Reminder to follow up on internship \\
%		\hline
	\end{tabular}
\end{table}

\newpage
\textbf{\large{Task 2: Processing Decisions}}

\begin{itemize}
	\item How should the materials will be processed according to GTD? 
\end{itemize}

\renewcommand{\arraystretch}{1.5} 
\begin{table}[htbp]
	\centering
	\begin{tabular}{|c|p{6cm}|p{8cm}|}
		\hline
		\textbf{Nr} & \textbf{Material} & \textbf{Decision} \\ \hline
		1  & WI-Projektarbeit: Info an Anna                    &  \\ \hline
		2  & Schlenkerla                                        &  \\ \hline
		3  & EidWI Case                                         &  \\ \hline
		4  & Bestätigung: Volleyballkurs                        &  \\ \hline
		5  & Rückmeldung: ERASMUS                               &  \\ \hline
		6  & Checkliste: Prüfungsvorbereitung                  &  \\ \hline
		7  & Buch ausgeliehen                                   &  \\ \hline
		8  & Auslandsstudienführer                              &  \\ \hline
		9  & FlexNow User’s Guide                               &  \\ \hline
		10 & Prüfungsanmeldung                                  &  \\ \hline
		11 & DAIMLER / Praktikumsnotiz                          &  \\ \hline
		12 & MITSloan Report: Digital Technology       &  \\ \hline
	\end{tabular}
\end{table}


\end{document}
